\section{Lecture 3}
\subsection{Kleene Fixed-Point Theorem}
\begin{definition}[$\bigsqcup$-continuous]
    Consider a complete lattice $(A, \sqsubseteq)$.
	A function $f \colon A \to A$ is $\bigsqcup$-continuous if, for all increasing sequences $x_0 \sqsubseteq x_1 \sqsubseteq x_2 \sqsubseteq x_2 \sqsubseteq \ldots$, we have
    \begin{align*}
        f \left( \bigsqcup \{ x_n : n \in \NN \} \right) = \bigsqcup \{ f(x_n) : n \in \NN \} \,.
    \end{align*}
\end{definition}


\begin{definition}[$\bigsqcap$-continuous]
    Consider a complete lattice $(A, \sqsubseteq)$.
	A function $f \colon A \to A$ is $\bigsqcap$-continuous if, for all increasing sequences $x_0 \sqsupseteq x_1 \sqsupseteq x_2 \sqsupseteq x_2 \sqsupseteq \ldots$, we have
 \begin{align*}
     	f \left( \bigsqcap \{ x_n : n \in \NN \} \right) = \bigsqcap \{ f(x_n) : n \in \NN \} \,.
 \end{align*}
\end{definition}

\begin{lemma}
    Both $\bigsqcup$-continuous and $\bigsqcap$-continuous imply monotonicity respectively. 
\end{lemma}



\begin{theorem}[Constructive Fixed-Point, Kleene's Fixed-Point]
    \label{thrm:constructive-fixedpoints}
    Consider a complete lattice $(A, \sqsubseteq)$. Then, for every $\bigsqcup$-continuous function $f \colon A \to A$, we have
    \[
    \lfp f = \bigsqcup \{ f^n(\bot) : n \in \mathbb{N} \},
    \]
    and for every $\bigsqcap$-continuous function $f \colon A \to A$, we have
    \[
    \gfp f = \bigsqcap \{ f^n(\top) : n \in \mathbb{N} \}.
    \]
\end{theorem}    


\begin{exercise}
    Prove the Constructive Fixed-Point Theorem (Theorem \ref{thrm:constructive-fixedpoints})
\end{exercise}
	

\subsection{Induction and Co-Induction}

\begin{definition}
\label{def:natural-numbers}
    Define $\NN$ as the smallest set X s.t.\ (i) $0 \in X$ and (ii) if $n \in X$, then $S n \in X$
\end{definition}

\begin{remark}
    In Definition \ref{def:natural-numbers}, we consider a universal set $U$ sufficiently big, the complete lattice $(2^U, \subseteq)$ and the function on sets given by $f(Y) \coloneqq \{0\} \cup \{ S n : n \in Y \}$. 
Then, $\lfp f = \NN$.	
\end{remark}



\begin{definition}
    Consider a finite alphabet $\Sigma$.
	Define $\Sigma^*$ as the smallest set $X$ such that (i) $\emptystring \in X$ and (ii) for all $a \in \Sigma$, we have $aX \subseteq X$.
\end{definition}

\begin{remark}
    Inductively defined sets are countable and consist of finite elements.
    They can be written as rules of the form
\begin{prooftree}
   \AxiomC{$x$}
   \UnaryInfC{$f(x)$}
\end{prooftree}
    expressing that if $x \in X$, then $f(x) \in X$. Moreover, they allow proof by induction. 
    \begin{quote}
         Consider proving that for all $x \in X$ we have $K(x)$. This can be proven by showing (i) $K(\bottom)$ and (ii) for all $x \in X$, if $K(x)$, then $K(f(x))$.
    \end{quote}
\end{remark}

\begin{definition}[Balanced binary sequences]
    Define the set $S$ as the largest set $X$ such that $X \subseteq 01X \cup 10X$.
\end{definition}


\begin{remark}
    In the definition of balanced binary sequences, we consider the complete lattice $(\mathcal{P}(\Sigma^\omega), \subseteq)$ and the function on sets given by $f(X) \coloneqq 01X \cup 10X$. Then, balanced binary sequences correspond to $\gfp f$.	
    
    Performing Kleene Fixed-Point iteration with $f(\cdot)$ on $\top$ yields the following initial elements:
    \begin{align*}
        \top &= \mathcal{U}, \\
        f^1(\top) &= f(\top) = 01\mathcal{U} \cup 10\mathcal{U}, \\
        f^2(\top) &= f(f(\top)) = 0101\mathcal{U} \cup 0110\mathcal{U} \cup 1001\mathcal{U} \cup 1010\mathcal{U}, \\
        &\ \vdots
    \end{align*}
\end{remark}


\begin{definition}[Interval {$[0, 1]$}]
    Define the set $S$ as the largest set $X$ such that $X \subseteq 0X \cup 1X \cup \ldots \cup 9X$.
\end{definition}


\begin{remark}
    Co-inductively defined sets are uncountable and consist of infinite elements.
    They can be written as rules of the form
    \begin{prooftree}
   \AxiomC{$x$}
   \UnaryInfC{$f(x)$}
\end{prooftree}
   expressing that for all $y \in X$, there exists $x$ such that $y = f(x)$ and $x \in X$. Moreover, they allow proof by co-induction.
    \begin{quote}
        Consider proving that for all $x$, if $K(x)$, then $x \in X$.
		This can be proven by showing (i) for all $x$ and $i$, if $K(f_i(x))$, then $K(x)$ \,, where $\{f_1, \ldots, f_n\}$ is the set of rules that define the set $X$.
    \end{quote}
\end{remark}




